\section{Introduction}
\label{sec:intro}

The measurement of \cPZ$/\cPgg^*$ (henceforth denoted by ``\cPZ'') or
\Wpm production in association with b quarks in proton-proton
collisions at the Large Hadron Collider (LHC) is relevant for various
experimental searches. In particular, these processes constitute a
background to standard model (SM) Higgs production associated with a
\cPZ~or a \Wpm boson, where the Higgs boson decays into a $b\bar{b}$
pair. The discovery by the ATLAS and Compact Muon Solenoid (CMS)
experiments of a neutral boson with a mass of about
$125\GeV$~\cite{Aad20121,:2012gu} motivates further studies to
establish its nature and determine the coupling of the new boson to b
quarks. Furthermore, different models based on an extension (minimal
or not) of the Higgs sector are being tested against the LHC data
through final states composed of lepton(s) and b-jet(s).  In this
context, a better understanding of the b-hadrons and/or associated
jets dynamics, as well as of the vector boson is required to refine
the background prediction and increase the sensitivity to new physics.

In addition, the study of the angular correlations for the collinear
production of b quarks, where large theoretical uncertainties remain
[ref?]  provides a strong lever arm on predictions from
theory. Tree-level calculations allowing for large numbers of extra
partons in the matrix elements (as initial- and final-state
radiations) are available. These are provided by
\MADGRAPH~\cite{Alwall:2007st,Alwall:2011uj}, 
{\ALPGEN~\cite{alpgen}, and \SHERPA~\cite{Gleisberg:2008ta},
in both the five- and four-flavour approaches, \ie by considering the
five or four lightest quark flavours in the proton parton distribution
function (PDF) sets. Next-to-leading-order (NLO) calculations have
been performed in both the five-flavour (\MCFM)~\cite{Campbell:2000bg}
and four-flavour~\cite{FebresCordero:2008ci,Cordero:2009kv}
approaches. A fully-automated NLO computation matched to a parton
shower simulation is implemented by the a\MCATNLO event
generator~\cite{Frederix:2011qg,Frixione:2002ik}. A detailed
discussion of b-quark production in the different calculation schemes
is available in Ref.~\cite{Maltoni:2012pa}.

From the experimental point of view, final states containing one or
two leptons (electrons or muons) and one ore more b-flavoured objects
are selected. Leptons from the vector-boson decays are expected to be
energetic and isolated, which provide a clean way of selecting the
events at the trigger level. The study of b-flavoured final states is
possible through two complementary techniques. First using the
standard jet-based b-tagging methods~\cite{Chatrchyan:2012jua}. These
techniques are dominantly considered in searches sensitive to the
emission of b jets. Second via the complementary approach that
identifies the b hadrons from displaced secondary vertices, which are
reconstructed from their charged decay products. This approach is
implemented in the inclusive secondary vertex finder
(IVF)~\cite{Khachatryan:2011wq}. The IVF exploits the excellent
tracking capabilities of the CMS detector and, being independent of
the jet reconstruction, extends the sensitivity to small angular
separations and softer b-hadron transverse momenta ($\pt$). Background
processes are mainly top processes (\ttbar and single top),
misidentification of light or charm-flavoured objects, dibosons, and
multijet production for the case of the \Wpm\ bosons.
 
In the past, the production of \cPZ+1 and 2 b
jets~\cite{Chatrchyan:2012vr,Chatrchyan:2014dha,Chatrchyan:2013zja,Aad:2011jn,Aad:2014dvb},
and \Wpm+ b jets~\cite{Chatrchyan:2013uza,Aad:2013vka} have been
studied by the ATLAS and CMS collaborations, at a centre-of-mass
energy $\sqrt{s}$ of 7 TeV using up to 5~fb$^{-1}$ of integrated
luminosity.

This article presents several studies. First, precise measurements,
not statistically limited, of the unbinned production cross section
for \cPZ+1 and 2 b-jets, and \Wpm+2 b-jets, are re-iterated at
$\sqrt{s}=8$ TeV. This is particularly important for the comparison
with the normalisation predictions provided at NLO.

Second, differential production cross sections are measured as a
function of several variables: $\pt^{\rm V}$ where V=\cPZ, \Wpm,
$\pt^{\cPqb_i}$ and $\eta^{\cPqb_i}$ where $i$ is denoting the list of
b jets sorted by transverse momentum, the mass of the b-jets system
$m^{\cPqb_1,\cPqb_2}$. These variables are considered in searches for
new physics states through channels involving b jets, and a comparison
between the data and predictions from theory is therefore crucial to
understand how well the predicted shapes match the reality.

Third, the production cross section of the processes where two b
hadrons are found in one single jet is provided, as well as the
angular correlations between the two b hadrons: $\Delta
R_{\mathrm{BB}}$ and $\Delta\phi_{\mathrm{BB}}$. The variable $\Delta
R_{\mathrm{BB}}$ is defined as $\Delta R_{\mathrm{BB}} = \sqrt{
  (\Delta \phi_{\mathrm{BB}})^2 + (\Delta \eta_{\mathrm{BB}})^2}$,
where $\Delta \phi_{\mathrm{BB}}$ and $\Delta \eta_{\mathrm{BB}}$ are
the azimuthal (in radians) and pseudorapidity separations. The
pseudorapidity is defined as $\eta= -\ln[\tan(\theta/2)]$, where
$\theta$ is the polar angle relative to the anticlockwise beam
direction. Besides helping to understand the contamination of V+2 b
hadrons inside one jet to V+1 b hadron in one jet, the $\Delta
R_{\mathrm{BB}}$ distribution constitutes a direct test of the
modeling of the different $\mathrm{pp}\rightarrow {\rm V}
\bbbar\mathrm{X}$ production modes. This quantity allows the
identification of the contribution from the $\cPq{i} \rightarrow {\rm
  V} \bbbar \mathrm{X}$ subprocesses (where $i=\cPq$, $\cPg$) for
which the scattering amplitude modelling is based on Feynman diagrams
with a final $\cPg \rightarrow \bbbar$ splitting. In particular, at
leading order, and assuming no selection on the transverse momentum of
the vector boson, this type of subprocess is dominant over the whole
$\Delta R_{\mathrm{BB}}$ range for $\Wpm b\overline{b}$ while it is
true only for small $\Delta R_{\mathrm{BB}}$ for $\cPZ b\overline{b}$
process. This is due to the dominant subprocesses with two
initial-state gluon splittings or where the \cPZ\ is radiated off a
final-state b-quark line. In these cases, the angular correlation
between the b quarks is much weaker. A leading-order diagram for both
$\cPq\cPq\rightarrow \cPZ b\overline{b}$ and $\cPq\cPq\rightarrow \Wpm
b\overline{b}$ is shown in Fig.~\ref{diag} (a), together with
diagrams representative of other $\mathrm{pp}\rightarrow\cPZ\bbbar$
production modes: emission of a \cPZ~boson from a b-quark line (b),
and b-quark fusion $\cPg\cPg \rightarrow \cPZ\bbbar$ (c).

\begin{figure}[!t]
	\begin{center}
	\includegraphics[width=0.5\textwidth]{figures_for_paper/croppedWZbb.pdf}
	\caption{Tree-level Feynman diagrams for (a) $\cPq\cPq
          \rightarrow {\rm V}\bbbar$ subprocess
          involving $\cPg\rightarrow \bbbar$
          splitting; (b)  $\cPq\cPaq \rightarrow \cPZ\bbbar$ with the
          emission of a $ \cPZ$~boson from a b quark; and (c) $\cPg\cPg \rightarrow \cPZ\bbbar$.}
\label{diag}	
	\end{center}
\end{figure}

For each measurement, the observed results are compared to several
theoretical predictions.


The sections are organized as follows: the description of the CMS
experiment is given in Section~\ref{sec:cms}; data and Monte-Carlo
samples are detailed in Section~\ref{sec:samples}; the objects
reconstruction and selection are presented in
Section~\ref{sec:eventrecoandselection}. The discussions on the
inclusive and differential cross-section measurements (based on b
jets) are found in Section~\ref{sec:bjetinclusivexsec} and
Section~\ref{sec:bjetdifferentialxsec}.  The measurement of the cross
section corresponding to the collinear production of b hadrons and the
discussion on angular correlation between the b hadrons are detailed
in Section~\ref{sec:collxsec}. The conclusions are presented in
Section~\ref{sec:conclusions}.

