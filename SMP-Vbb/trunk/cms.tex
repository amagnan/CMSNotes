\section{CMS detector}
\label{sec:cms}


The central feature of the CMS apparatus is a superconducting solenoid
of 6\unit{m} internal diameter, providing a magnetic field of
3.8\unit{T}. Within the superconducting solenoid volume are a silicon
pixel and strip tracker, a lead tungstate crystal electromagnetic
calorimeter (ECAL), and a brass/scintillator hadron calorimeter
(HCAL), each composed of a barrel and two endcap sections. Muons are
measured in gas-ionization detectors embedded in the steel flux-return
yoke outside the solenoid. Extensive forward calorimetry complements
the coverage provided by the barrel and endcap detectors.

The most relevant detector element for the identification of b jets is
the tracking system. The inner tracker consists of 1440 silicon pixel
and 15 148 silicon strip detector modules. It measures charged
particles up to a pseudorapidity of $|\eta| < 2.5$. The pixel modules
are arranged in three cylindrical layers in the central part of CMS
and two endcap disks on each side of the interaction point. The sili-
con strip detector comprises two cylindrical barrel detectors with a
total of 10 layers and two endcap systems with a total of 12 layers at
each end of CMS. The tracking system provides an impact parameter (IP)
resolution of about 15 $\mu$m at a \pt of 100 GeV. In comparison
typical IP values for tracks from b-hadron decays are at the level of
a few 100 $\mu$m. 

Muons are measured and identified in detection layers that use three
technologies: drift tubes, cathode-strip chambers, and resistive-plate
chambers. The muon system covers the pseudorapidity range $|\eta| <
2.4$. The combination of the muon and tracking systems yields muon
candidates of high purity with a \pt resolution between 1 and 5\%, for
\pt values up to 1 TeV. The ECAL energy resolution for electrons with
$\ET {\approx} 45$\GeV from $\Z \rightarrow \Pe \Pe$ decays is better
than 2\% in the central region of the ECAL barrel $(\abs{\eta} <
0.8)$, and is between 2\% and 5\% elsewhere. For low-bremsstrahlung
electrons, where 94\% or more of their energy is contained within a $3
\times 3$ array of crystals, the energy resolution improves to 1.5\%
for $\abs{\eta} < 0.8$~\cite{Chatrchyan:2013dga}.

A more detailed description of the CMS detector, together with a
definition of the coordinate system used and the relevant kinematic
variables, can be found in Ref.~\cite{Chatrchyan:2008zzk}.

\section{Data and Monte-Carlo samples}
\label{sec:samples}

The data were taken in 2012. Beam conditions changed during the year
in order to provide the highest instantaneous luminosity possible,
which meant that the average number of secondary minimum bias
interactions (pileup) varied from about 10 to 35. Only luminosity
sections for which all subdetectors were fully operational are
considered for analysis. The total analysed integrated luminosity
amounts to $19.8 \pm 0.5$\,fb$^{-1}$~\cite{CMS-PAS-LUM-13-001}.

Events were triggered by single lepton (W) and dilepton triggers,
choosing the loosest unprescaled versions available. All triggers used
here are selecting leptons above given p$_T$ thresholds and passing
loose quality and isolation criteria. The single electron (muon)
trigger has a p$_T$ threshold on the electron (muon) of 27
(24)\,GeV). The single muon trigger also has a requirement on the
pseudorapidity $|\eta|<2.1$. The dielectron and dimuon triggers have
p$_T$ thresholds of 17 and 8\,GeV on the leading and subleading
leptons.

For \ttbar, W+jets and Z+jets processes, the Monte Carlo (MC)
simulation of events is done using the \madgraph
v5~\cite{Alwall:2011uj} matrix-element (ME) generator interfaced with
\pythia v6.4~\cite{Sjostrand:2006za} for the parton shower (PS), using
the Z2 tune~\cite{Khachatryan:2010nk}. Events are generated with up to
4 partons at tree-level. The different parton multiplicities are
merged into a single sample. The cteq6l1 PDF set is
used~\cite{Pumplin:2002vw}. Normalisation and factorisation scales are
set to XXX. The k$_T$-MLM ME-PS matching mechanism is used with a
scale set to 20 GeV. V+jets (\ttbar) samples are normalised to
inclusive NNLO cross sections obtained with FEWZ~\cite{Gavin:2010az} (

For single top processes, ...

Diboson events (WW, WZ and ZZ) are generated using ...

Alternative MC are considered in order to extract model uncertainties
(see Section~\ref{sec:syst})....
