\section{Setup}
\label{sec:setup}

Two different setups are used, in order to systematically provide an
independant cross check of the results:
\begin{itemize}
\item standalone simulation of a detector with a transverse size of $20 \times 20$cm$^2$ and 30 layers.
\item full geometry of the final detector in CMSSW.
\end{itemize}

For the standalone simulation, two different analyses are also run in
parallel and cross-checking each others systematically.

\subsection{The standalone simulation setup}
\label{sec:standalone}

This setup is used for three different purposes: provide a quick
handle to optimise the design parameters (see
section~\ref{sec:optim}), extract performance results in ideal
conditions, and cross-check the performances in pileup conditions with
the CMSSW setup. The code is available in git under the following
repository: \url{https://github.com/pfs/PFCal/}.

\subsubsection{Geant4 setup}

The detector construction is made of 30 sampling sections, each with
the following parameters:
\begin{table}{h!}
\caption{\label{tab:samplSec} Thicknesses (in mm) for the different materials, per layer}
\begin{tabular}{|l|c|c|c|c|c|}
\hline
Layers & Pb & Cu & Si & PCB & Air \\
\hline
1-10  & 1.63 & 3 & 0.2 & 1 & 2 \\
11-20 & 3.32 & 3 & 0.2 & 1 & 2 \\
21-30 & 5.56 & 3 & 0.2 & 1 & 2 \\
\hline
\end{tabular}
\end{table}

Figure~\ref{fig:g4vis} shows an event display of a 50 GeV electron
shower in two different geometries: a uniform 26 layers of 1 X$_0$
each (left) and the baseline geometry detailed in
table~\ref{tab:samplSec}.

\begin{figure}[h!]
  \begin{center}
    \includegraphics[width=\cmsFigWidth]{figures/e_50GeV_uniform_26x0.png}
    \includegraphics[width=\cmsFigWidth]{figures/e_50GeV_concept_v3.png}
    \caption{Event displays of a 50 GeV electron showers in the
      standalone simulation. Left: uniform 26 layers of 1 X$_0$
      each. Right: 30 layers in three blocks of different
      thicknesses. Electron deposits are shown in blue. Photons have
      been hidden for better visibility.}
    \label{fig:g4vis}
  \end{center}
\end{figure}

The simulation uses the QGSP\_BERT physics list, with the default cuts
for electrons, positrons and photons reduced to 10\,$\mu$m in the
silicon. The default elsewhere is 700\,$\mu$m. With 700\,$\mu$m in Si,
a 420 keV (5.8 keV) electron (photon) would deposit all its energy in
one step. With the reduced 10\,$\mu$m, the energy threshold is 32 keV
for electrons, and 990 eV for photons.

%%
%%
%%
\subsubsection{Energy resolution}
\label{subsubsec:energyresol}

Electromagnetic energy resolution is estimated with single electron events.
The electrons are made to incide perpendicularly to the face of the calorimeter and the energy deposited by the electromagnetic shower is counted in each of the Si layers.
