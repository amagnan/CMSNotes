%
%
%
\clearpage
\section{Introduction}
\label{sec:intro}

It is now established that the CMS endcap calorimeters will have to be replaced for the high
luminosity phase of LHC (HL-LHC). During the anticipated lifetime of
the HL-LHC it is expected
that an integrated luminosity of 3000\fbinv will be collected.
Any calorimeter that is placed in the endcap region of CMS will need to be able to contend with both the very high level of pile-up
events and the very high level of ionizing and neutron irradiation that will be associated with the
high luminosity running. The design of the calorimeter needs to take these factors into account
and below we explore an option of a sampling calorimeter based on silicon sensors for the active
material with high granularity both longitudinally and laterally - HGCal.
More details on the HGCal project can be found in~\cite{HGCal}.

In this manuscript we describe the studies carried for the optimization of
the electromagnetic calorimeter sub-detector of HGCal (HGCal-EE) which is
required to have good electormagnetic resolution as a crucial input to
the full jet reconstruction in the forward region.


Our studies are presented as follows:
in Sec.~\ref{sec:setup} the geometry of the baseline design is
detailed using a stand-alone \GEANT 4~\cite{1610988,Agostinelli:2002hh}
simulation and a full integration in a upgrade scenario within \CMSSW;
the generic properties of the electromagnetic showers in the HGCal-EE baseline geometry
are discussed in Sec.~\ref{sec:emshowerproperties};
the impact on the energy resolution resulting from using different detector
configurations,
including different silicon and sampling thicknesses or absorber
materials is studied in Sec.~\ref{sec:optim};
a digitisation procedure is explained in section~\ref{sec:digi}. 
We conclude our manuscript by presenting detailed
studies of the performances, in ideal conditions and with pileup, in Sec.~\ref{sec:perf}.
